\documentclass[../main.tex]{subfiles}
\graphicspath{{\subfix{../images/}}}
\begin{document}

When reading Chapter 12: Implementing a Multilayer Artificial Neural Perceptron
from Scractch, Dr. Raschka used the following formula to compute the partial 
derivative of the loss function with respect to the all the weight in $W^{(out)}$
when backpropagating:

\[\frac{\partial J(W)}{\partial W_{i,j}^{(out)}} = (A^{(h)})^{T} \delta^{(out)}\]

I did not understand it at first, but after many trials I now do. In this document,
I want to walk you through how I computed the partial derivative of the loss function
with respect to all the weights in $W^{(out)}$, and got to the result Sebastian presented in
his book.

\vspace{5mm} %5mm vertical space

In chapter 12, Dr. Raschka used the following as the cost function:

\[J(w) = -\sum_{i = 1}^{n}\sum_{j=1}^{t} y_j^{[i]} log(a_j^{[i]}) + (1 - y_j^{[i]})log(1 - a_j^{[i]})\]

\pagebreak

In $J(W)$:
\vspace{5mm} %5mm vertical space
\begin{itemize}
    \item $n$ is the number of training examples.
    \item Superscript $[i]$ refers to a specific training example.
    \item $t$ is the number of activations in the last (outptut) layer.
    \item Subscript $j$ refers to a specific output neuron.
    \item $J(W)$ is the `loss' (or `error') value. It tells how well the network is performing. Given
    a training example, the function uses the activation values in the output layer and returns
    the error.
\end{itemize}

$a_j^{[i]}$ refers to \emph{the activation value of the $j^{th}$ neuron in
the output layer, after the $i^{th}$ example is forward propagated through the 
network}. 

\vspace{5mm} %5mm vertical space

$y_j^{[i]}$ on the other hand, is \emph{the target value for the
neuron $j$ in the output layer, given a training example $i$}. 
In other words, during training ... we may forward propagate an example $i$ through the network 
and obtain different activations values in the output layer; we want activation $a_j^{[i]}$ to get as 
close as possible to $y_j^{[i]}$. 

\vspace{5mm} %5mm vertical space

\end{document}