\documentclass[../main.tex]{subfiles}
\graphicspath{{\subfix{../images/}}}
\begin{document}

The Jacobian matrix we wrote earlier contains the gradients of all the
weights in $W^{(out)}$ for \emph{only one} example. If a second training example goes 
through the network, a new loss value is computed...  and a new gradient is computed
for \emph{each} weight in $W^{(out)}$ during backpropagation. The new gradient of \emph{each} weight
is \textbf{added} to the gradients of the same weight when backpropagation happened for the previous example.
It's like they accummulate.

\vspace{5mm} %5mm vertical space

It looks like this matrix form, when $n$ training example go through the network:

\[
    \begin{bmatrix}
        \sum_{i=1}^{n} a_1^{[i]}\delta_1^{[i]}  & \sum_{i=1}^{n} a_1^{[i]}\delta_2^{[i]} & \dots    & \sum_{i=1}^{n} a_1^{[i]}\delta_t^{[i]} \\
        \sum_{i=1}^{n} a_2^{[i]}\delta_1^{[i]}  & \sum_{i=1}^{n} a_2^{[i]}\delta_2^{[i]} & \dots    & \sum_{i=1}^{n} a_2^{[i]}\delta_t^{[i]} \\
        \vdots                                  & \vdots                                 & \ddots   & \vdots                                 \\
        \sum_{i=1}^{n} a_d^{[i]}\delta_1^{[i]}  & \sum_{i=1}^{n} a_d^{[i]}\delta_2^{[i]} & \dots    & \sum_{i=1}^{n} a_d^{[i]}\delta_t^{[i]}
    \end{bmatrix}
\]

\vspace{5mm} %5mm vertical space

The above matrix can be deconstructed in two seperate matrices giving the following

\vspace{5mm} %5mm vertical space

\[
    \begin{bmatrix}
        a_1^{[1]}   & a_1^{[2]} & \dots  & a_1^{[n]} \\
        a_2^{[1]}   & a_2^{[2]} & \dots  & a_2^{[n]} \\
        \vdots      & \vdots    & \ddots & \vdots    \\
        a_d^{[1]}   & a_d^{[2]} & \dots  & a_d^{[n]}
    \end{bmatrix}
    \begin{bmatrix}
        \delta_1^{[1]}   & \delta_2^{[1]} & \dots  & \delta_t^{[1]} \\
        \delta_1^{[2]}   & \delta_2^{[2]} & \dots  & \delta_t^{[2]} \\
        \vdots      & \vdots    & \ddots & \vdots    \\
        \delta_1^{[n]}   & \delta_2^{[n]} & \dots  & \delta_t^{[n]} 
    \end{bmatrix}
    =
    (A^{(h)})^T \delta^{(out)}
\]

\emph{Note: the superscript $[i]$ refers to a specific training example}.

\vspace{5mm} %5mm vertical space

This is the formula Dr. Raschka used in the book. I hope me walking you through
my reasonning helped you reach a better understanding.
\end{document}